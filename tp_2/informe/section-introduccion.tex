%!TEX TS-program = xelatex
%!TEX encoding = UTF-8 Unicode

El cerebro es el órgano encargado de controlar y regular de forma centralizada las funciones del cuerpo. Éste consta de regiones diferenciadas asociadas a determinadas facultades específicas como las mentales, comportamientos u operaciones cognitivas, que trabajan de forma sincronizada e integrándose para controlar todo nuestro organismo. Por ello, se trata de un órgano de gran importancia y cuya relevancia ha dado lugar a numerosos estudios tanto anatómicos como fisiológicos, así como de neuroimagen, para entender su funcionamiento, tanto en condiciones normales como en condiciones patológicas, e incluso en diferentes estados de conciencia (dormidos o despiertos).\\
En la última década venimos asistiendo al fuerte auge de lo que se está dando en llamar la \textbf{Neurociencia de redes}, esto es, el estudio de la conectividad de redes cerebrales, mediante técnicas procedentes de la teoría de grafos.\\
La teoría de redes complejas es una herramienta muy potente que permite estudiar el cerebro desde el punto de vista de una red compleja, de forma que los nodos representan neuronas individuales o grupos neuronales y los enlaces entre éstos pueden representar tractos neuronales (conectividad estructural), relaciones de dependencia estadística (conectividad funcional) o efectos causales que una región ejerce sobre otra (conectividad efectiva).
Este trabajo tiene como antecedente el realizado por Tagliazucchi y colaboradores (2013), donde se busca relacionar cambios en la modularidad de las redes construidas a partir de la señal de resonancia magnética funcional con los distintos estadíos del sueño. \\
Nuestro objetivo es analizar y explorar los cambios en estas redes en función de la profundidad del sueño (N1, N2 y N3) y también establecer comparaciones contra la actividad cerebral cuando un individuo está despierto. Cabe destacar, que en el trabajo realizado por Tagliazucchi su estudio se basó en 63 sujetos y el presente trabajo será realizado analizando a 18 de estos, apoyándonos en la teoría de grafos, en técnicas de Clustering y métodos como el de Louvain. 